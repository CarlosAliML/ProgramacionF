% Ejemplo de documento LaTeX
% Tipo de documento y tamaño de letra
\documentclass[12pt]{article}










\usepackage[spanish]{babel}
\usepackage{longtable} 
\selectlanguage{spanish}
\usepackage[utf8]{inputenc}








% EL titulo, autor y fecha del documento
\title{Guía de comandos para Bash}
\author{Carlos Medina}
\date{9-2-15}








% Aqui comienza el cuerpo del documento
\begin{document}
% Construye el título
\maketitle
















El siguiente texto es una pequeña guia para la terminal de Linux, así como algunos comandos básicos para usarlo y entender mejor la utilidad de este sistema operativo.








A continucación le mostraremos una lista de comandos y su función.










\begin{tabular}{|c|p{3cm}|p{10cm}|p{3cm}|}
    \cline{1-3}
    \multicolumn{3}{|c|}{Comandos Linux}\\
    \cline{1-3}
    Comando & Ejemplo & Descripción de comando\\
    \cline{1-3}
    \centering
    man & $man ls$ & Te describe para qué sirve el comando ls. te muestra un manual y la función del comando que escribas después de éste.\\
    \hline
    cat & $cat prueba.txt$ &Nos permite visualizar el contenido de un archivo de texto sin la necesidad de un editor, debemos mencionarlo junto al archivo que deseamos visualizar.\\
    \hline
    ls & $ls -al$ &Permite listar el contenido de un directorio o fichero. Puedes usar -a para mostrar los archivos ocultos y -l para mostrar los usuarios, permisos y la fecha de los archivos. \\
    \hline
    cd & cd "/"home"/"ejercicios & Viene de "cambiar directorio" Te permite acceder a una ruta distinta de la que te encuentras. En el ejemplo pueden ver si estas en el directorio /home y deseas acceder a /home/ejercicios\\
    \hline 
    touch & touch /home/prueba1.txt & Crea un archivo vacío, si el archivo existe actualiza la hora de modificación. Para crear el archivo prueba1.txt en /home\\
    \hline 
    Mkdir & mkdir /home/ejercicios & Viene de "make directory" o "crear directorio", crea un directorio nuevo tomando en cuenta la ubicación actual. Por ejemplo, si estas en /home y deseas crear el directorio ejercicios\\
    \hline 
  \end{tabular}
  
\begin{tabular}{|c|p{3cm}|p{10cm}|p{3cm}|}
    \cline{1-3}
    \cline{1-3}
    Comando & Ejemplo & Descripción de comando\\
    \cline{1-3}
    \centering
     \hline 
    cp & cp /home/prueba.txt /home/respaldo/prueba.txt & Viene de "copy" o "copiar", copia un archivo o directorio origen a un archivo o directorio destino. Por ejemplo, para copiar el archivo prueba.txt ubicado en /home a un directorio de respaldo\\
    \hline 
    mv & mv /home/prueba.txt /home/respaldos/prueba2.txt & Viene de "move" o "mover", mueve un archivo a una ruta específica, y a diferencia de cp, lo elimina del origen finalizada la operación.\\
    \hline 
    rm & rm /home/prueba.txt & Viene de "remove" o "remover", es el comando necesario para borrar un archivo o directorio. Para borrar el archivo prueba.txt ubicado en $/home$\\
    \hline 
    -fr & rm -fr /home/respaldos & La opción -r borra todos los archivos y directorios de forma recursiva. Por otra parte, -f borra todo sin pedir confirmación. Estas opciones se usan en el comando rm y pueden combinarse causando un borrado recursivo y sin confirmación del directorio que se especifique. Para realizar esto en el directorio respaldos ubicado en el /home\\
    \hline 
    * & $*a$   ~Para los archivos que terminen en "a" y $a*$   ~Para los archivos que empiezen en "a" & Nos permite buscar en base al termino o inicio de un archivo con una letra o numero dependiendo de donde situas el asterisco.\\
    \hline 
    clear & clear & Viene de "limpiar", y es un sencillo comando que limpiara nuestra terminal por completo dejándola como recién abierta.\\
    \hline 
    echo & echo hola & Sirve para repetir las palabras, o bien, para repetir tu texto en un archivo de texto.\\
    \hline 
    pwd& pwd& Te muestra en qué directorio te encuentras, es decir, en qué directorio estás realizando las acciónes y comandos.\\
    \hline 
  \end{tabular}
  
  